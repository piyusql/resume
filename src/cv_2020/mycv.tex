%%%%%%%%%%%%%%%%%%%%%%%%%%%%%%%%%%%%%%%%%
% Freeman Curriculum Vitae
% XeLaTeX Template
% Version 2.0 (19/3/2018)
%
% This template originates from:
% http://www.LaTeXTemplates.com
%
% Authors:
% Vel (vel@LaTeXTemplates.com)
% Alessandro Plasmati
%
% License:
% CC BY-NC-SA 3.0 (http://creativecommons.org/licenses/by-nc-sa/3.0/)
%
%!TEX program = xelatex
% NOTICE: This template must be compiled with XeLaTeX, the line above should
% ensure this happens automatically but if it doesn't you will need to specify 
% XeLaTeX as the engine in your editor or script
% 
%%%%%%%%%%%%%%%%%%%%%%%%%%%%%%%%%%%%%%%%%

%----------------------------------------------------------------------------------------
%	PACKAGES AND OTHER DOCUMENT CONFIGURATIONS
%----------------------------------------------------------------------------------------

\documentclass[10pt]{article} % Font size, can be: 10pt, 11pt or 12pt

\input{structure.tex} % Include the file that specifies the document structure

% Headers and footers can be added with the \lhead{} \rhead{} \lfoot{} \rfoot{} commands
% Example right footer:
%\rfoot{\color{headings}{\sffamily Last update: \today. Typeset with Xe\LaTeX}}

%----------------------------------------------------------------------------------------

\begin{document}

\begin{paracol}{2} % Begin the multi-column environment

%----------------------------------------------------------------------------------------
%	NAME AND CURRICULUM VITAE TEXT
%----------------------------------------------------------------------------------------

\parbox[top][0.12\textheight][c]{\linewidth}{ % Parbox to hold the author name and CV text; fixed height to match the coloured box to the right, centred vertically and full line width
	\vspace{-0.04\textheight} % Reduce whitespace above the parbox to separate it from the main content
	\centering % Centre text
	{\sffamily\Huge Piyus Gupta}\\\medskip % Your name
	{\Huge\color{headings}\cvtextfont Sr MTS at VMware}
}

%----------------------------------------------------------------------------------------
%	MAJOR RESEARCH PROJECT
%----------------------------------------------------------------------------------------

\section{About Me}

{\raggedright\textbf{``Piyus is equipped with 12+ yrs of software development experience in building products in startups, the travel industry, and VMware. With a sound background in mathematics, he loves to play with data. Most of his time was spent writing back-end applications, providing scalable and high availability solutions to the distributed applications, and writing automated test suites and continuous integration to bind the application with a scale of robustness."}\\\medskip}

    {The most amazing thing I have learned is the way to set baseline of the product and continuously make efforts to better it every day.}
    {I learned breaking problems in smaller pieces and follow the divide and conquer approach to build great scale products.}
    {I believe in action over long-winded discussions. I listen to everyone's viewpoints and use my judgement to immediately act based on consensus to achieve goals quickly and efficiently.}

\medskip % Extra whitespace before the next section

%----------------------------------------------------------------------------------------
%	WORK EXPERIENCE
%----------------------------------------------------------------------------------------

\section{Recent Work Experience}

% Blank \workposition command to add another job:

%\workposition{} % Duration
%{} % FT/PT (full time or part time)
%{} % Employer
%{} % Job title
%{} % Description

% All 5 parameters must be supplied but any can be empty if you don't need them

%------------------------------------------------

\workposition{Current, from Aug 2015} % Duration
{} % FT/PT (full time or part time)
{VMware Software India Pvt Ltd} % Employer
{Sr Member of Technical Staff} % Job title
    {As part of this role, I am focused on architecture, writing designs and being the hands-on helping hand to the team.}
    {I have spent most of my time on handling postgres at scale and the microservice architecture in and around our cloud native apps developed to run our giant build system.}
    {I did split several monolithic giant applications and turned it into microservices. Used Docker and Kubernetes to host on the cloud with easy deployment and horizontal scalability.}
    {Wrote pipeline script in Jenkins and Groovy to enable continuous integration which verifies our code before submitting.}
    {Introduced a test-driven development approach that helped to set a baseline for our product feature before heading towards production and then later gauge the improvement based on that.} % Description

%------------------------------------------------

\workposition{Sep 2012 -- Aug 2015} % Duration
{} % FT/PT (full time or part time)
{HTMedia Ltd} % Employer
{Tech Lead} % Job title
{This position involved utilising the immense resources of python community and django third party batteries. The transition required an initial learning curve in web scaling, full text search, browser networking, prevention of sql injections and operating experimental infrastructure. Helping a team of six to eight developers improved my thought process and ability to handle big projects.}  % Description

%------------------------------------------------

\workposition{Nov 2009 -- Aug 2012} % Duration
{} % FT/PT (full time or part time)
{Knowlarity Communications Pvt Ltd} % Employer
{Product Development Engineer} % Job title
{In this particular job at seed startup, I learned most of the experimental methods, failing fast and build fast, improvement ideas and being passionate towards work.}
{To work in an enviroment where deadline was important even without compromising quality.} % Description

%------------------------------------------------
\vspace{-\baselineskip}\medskip % Standardise the whitespace after this section and before the next (the custom command adds too much otherwise)
%------------------------------------------------

\medskip % Extra whitespace before the next section

%----------------------------------------------------------------------------------------

\switchcolumn % Switch to the next paracol column

%----------------------------------------------------------------------------------------
%	COLOURED CONTACT DETAILS BOX
%----------------------------------------------------------------------------------------

\parbox[top][0.12\textheight][c]{\linewidth}{ % Parbox to hold the colour box; fixed height to match the name/CV text to the left, centred vertically and full line width
	\vspace{-0.04\textheight} % Reduce whitespace above the parbox to separate it from the main content
	\colorbox{shade}{ % Create the coloured box
		\begin{supertabular}{p{0.05\linewidth}|p{0.775\linewidth}} % Start a table with two columns, the table will ensure everything is aligned
            \raisebox{-1pt}{\faHome} & Electronic City, Bangalore, IN \\ % Address
			\raisebox{-1pt}{\faPhone} & +91-9953-247-269 \\ % Phone number
			\raisebox{0pt}{\small\faEnvelope} & \href{mailto:piyusgupta01@gmail.com}{piyusgupta01@gmail.com} \\ % Email address
			%\raisebox{-1pt}{\small\faDesktop} & \href{https://www.tracearound.com}{https://www.tracearound.com} \\ % Website
			\raisebox{-1pt}{\faGithub} & \href{https://github.com/piyusgupta}{https://github.com/piyusgupta} \\ % GitHub profile
			\raisebox{-1pt}{\faLinkedinSquare} & \href{https://www.linkedin.com/in/piyusgupta}{https://www.linkedin.com/in/piyusgupta} \\ % LinkedIn profile
			% See fontawesome.pdf in the fonts folder for all icons you can use
		\end{supertabular}
	}
}

%----------------------------------------------------------------------------------------
%	EDUCATION
%----------------------------------------------------------------------------------------

\section{Education} 

% Blank \educationentry{} command to add another degree:

%\educationentry{} % Duration
%{} % Degree
%{} % Honours, achievements or distinctions (e.g. first class honours)
%{} % Department
%{} % Institution

% All 5 parameters must be supplied but any can be empty if you don't need them

%------------------------------------------------

\begin{supertabular}{rl} % Start a table with two columns, the table will ensure everything is aligned

	%------------------------------------------------
	
	\educationentry{2003 -- 2007} % Duration
	{Bachelor of Technology} % Degree
	{} % Honours, achievements or distinctions (e.g. first class honours)
	{Computer Science and Engineering} % Department
    {Dr K N Modi Institute of Engg \& Technology} % Institution
	
	%------------------------------------------------
	
\end{supertabular}

%----------------------------------------------------------------------------------------
%	COMPUTER SKILLS
%----------------------------------------------------------------------------------------

\section{Skills and Expertise} 

% Example \tableentry{} command to add another line:

%\tableentry{Heading}{Content}{spaceafter}

% All 3 parameters must be supplied but any can be empty if you don't need them
% A "spaceafter" value in the third parameter will add some vertical space -- this is to be used between headings

%------------------------------------------------

\begin{supertabular}{rl} % Start a table with two columns, the table will ensure everything is aligned
	
	%------------------------------------------------
	
    \tableentry{Languages}{Python, Golang}{spaceafter}
	
	%------------------------------------------------
	
	\tableentry{Frameworks}{Django, REST APIs}{spaceafter}
	
	%------------------------------------------------
	
	\tableentry{Storage}{PostgreSQL, MySQL, Redis-Cache}{spaceafter}
	
	%------------------------------------------------
	
	\tableentry{Tools}{Git, Perforce, Reviewboard, \LaTeX}{spaceafter}
	
	%------------------------------------------------
	
    \tableentry{Platforms}{Docker, Kubernetes, Linux, AWS}{spaceafter}
	
	%------------------------------------------------
	
\end{supertabular}

%----------------------------------------------------------------------------------------
%	PROJECT DESCRIPTION
%----------------------------------------------------------------------------------------

\section{Projects}

% Example \longformdescription{} command to add another section:

%\longformdescription{Heading}{Description}

%------------------------------------------------

\longformdescription{Distributed Architecture at Scale}
    {Made improvements to our product which spawns and deletes 150,000+ VMs every day.}
    {Redis-cache search to enable user a smooth text search experience.}

\longformdescription{Horizontal scalibility}
    {Split several monolithic giant applications into microservices.}
    {Used Docker, Kubernetes and Helm charts to host on cloud with easy deployment and horizontal scalability.}

\longformdescription{Developer Productivity}
{Introduced a test-driven development approach that helped set a baseline for our product feature before heading towards production and then later gauges the improvement based on that.}
{Wrote a pipeline script in Jenkins and Groovy to enable continuous integration wich verifies your code before submitting.}

\longformdescription{Development}
{Wrote reporting SQL queries and designed beautiful charts using Postgres, Python, and Highcharts.}
{Improved the time to serve page bytes and rendering time in browsers using Steve Souders' high-performance websites technique. Improved the overall speed by 250\%.}
{Created RESTful APIs using the Knowlus platform to make bulk calls and reporting—2 million calls per day.}
{Integrated GDS for air ticketing e.g. Amadeus and Hermes, for a web panel for booking air tickets and hotels by agents across India 15k tickets per day back in 2009.}

\longformdescription{Passionate}{I have been interested in mathematics and contributing to opensource using my skills, a recent initiative.} {\faGithub}\href{https://github.com/piyusgupta/pgtuner}{pgtuner}
\medskip % Extra whitespace before the next section

%----------------------------------------------------------------------------------------

\end{paracol}

%----------------------------------------------------------------------------------------

\end{document}
