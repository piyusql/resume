%%%%%%%%%%%%%%%%%%%%%%%%%%%%%%%%%%%%%%%%%
% Freeman Curriculum Vitae
% Structure Specification File
% Version 1.0 (19/3/2018)
%
% This template originates from:
% http://www.LaTeXTemplates.com
%
% Authors:
% Vel (vel@LaTeXTemplates.com)
% Alessandro Plasmati
%
% License:
% CC BY-NC-SA 3.0 (http://creativecommons.org/licenses/by-nc-sa/3.0/)
% 
%%%%%%%%%%%%%%%%%%%%%%%%%%%%%%%%%%%%%%%%%

%----------------------------------------------------------------------------------------
%	PACKAGES AND OTHER DOCUMENT CONFIGURATIONS
%----------------------------------------------------------------------------------------

\usepackage{etoolbox} % Required for conditional statements

\setlength\parindent{0pt} % Stop paragraph indentation

\usepackage{supertabular} % Required for the supertabular environment which allows tables to span multiple pages so sections with tables correctly wrap across pages

%----------------------------------------------------------------------------------------
%	DOCUMENT MARGINS
%----------------------------------------------------------------------------------------

\usepackage{geometry} % Required for adjusting page dimensions and margins

\geometry{
	hmargin=1.1cm, % Horizontal margin
	vmargin=1.25cm, % Vertical margin
	a4paper, % Paper size, change to letterpaper for US letter size
	%showframe, % Uncomment to show how the type block is set on the page -- typically for debugging
}

%----------------------------------------------------------------------------------------
%	COLUMN LAYOUT
%----------------------------------------------------------------------------------------

\usepackage{paracol} % Required for creating multi-column layouts that can span pages automatically

\columnratio{0.56,0.44} % The relative ratios of the two columns in the CV

\setlength\columnsep{0.05\textwidth} % Specify the amount of space between the columns

%----------------------------------------------------------------------------------------
%	FONTS
%----------------------------------------------------------------------------------------

\usepackage{fontspec} % Required for specifying custom fonts under XeLaTeX

\setmainfont{EBGaramond}[ % Make the default font EBGaramond
Path=fonts/, % The font is provided with the template in the fonts folder
UprightFont=*-Regular.ttf,
BoldFont=*-Bold.ttf,
BoldItalicFont=*-BoldItalic.ttf,
ItalicFont=*-Italic.ttf,
SmallCapsFont=*-SC.ttf
]

\newfontfamily\cvtextfont[Path=fonts/]{freebooterscript} % Create a new font family for the cursive font Freebooter Script, provided with the template in the fonts folder

\newfontfamily{\FA}[Path=fonts/]{FontAwesome} % Create a new font family for FontAwesome icons, provided with the template in the fonts folder
\input{fonts/fontawesomesymbols-xeluatex.tex} % Load a file to create shortcuts to the icons, see icon examples and their shortcuts in fontawesome.pdf in the fonts folder

\usepackage[sf,scale=0.95]{libertine} % Load Libertine as a \sffamily font for sans serif titles

%----------------------------------------------------------------------------------------
%	COLOURS AND LINKS
%----------------------------------------------------------------------------------------

\usepackage[usenames,svgnames]{xcolor} % Allows the definition and use of custom colours

\definecolor{text}{HTML}{2b2b2b} % Main document font colour, off-black
\definecolor{headings}{HTML}{701112} % Dark red colour for headings
\definecolor{shade}{HTML}{F5DD9D} % Peach colour for the contact information box
\definecolor{linkcolor}{HTML}{641c1d} % 25% desaturated headings colour for links
% Other colour options: shade=B9D7D9 and linkcolor=A40000; shade=D4D7FE and linkcolor=FF0080

% For preset colours that can be used by their names without having to define them, see: https://www.latextemplates.com/svgnames-colors

\color{text} % Set the default text colour for the whole document to the colour defined as 'text' above

%------------------------------------------------

\usepackage{hyperref} % Required for links

\hypersetup{colorlinks, breaklinks, urlcolor=linkcolor, linkcolor=linkcolor} % Set up links and their colours


%----------------------------------------------------------------------------------------
%	HEADERS & FOOTERS
%----------------------------------------------------------------------------------------

\usepackage{fancyhdr} % Required for customising headers and footers

\pagestyle{fancy} % Enable custom headers and footers

\fancyhf{} % This suppresses all headers and footers by default, add headers and footers in the template file as per the example

\renewcommand{\headrulewidth}{0pt} % Remove the default rule under the header

%----------------------------------------------------------------------------------------
%	SECTIONS
%----------------------------------------------------------------------------------------

\usepackage[nobottomtitles*]{titlesec} % Required for modifying sections, the nobottomtitles* is required for pushing section titles to the next page when they are close to the bottom of the page

\renewcommand{\bottomtitlespace}{0.1\textheight} % Modify the minimal space required from the bottom margin not to move the title to the next page

\titleformat{\section}{\color{headings}\scshape\LARGE\raggedright}{}{0em}{}[\color{black}\titlerule] % Define the \section format

\titlespacing{\section}{0pt}{0pt}{8pt} % Spacing around section titles, the order is: left, before and after

%----------------------------------------------------------------------------------------
%	CUSTOM COMMANDS
%----------------------------------------------------------------------------------------

% Command for entering a new work position
\newcommand{\workposition}[5]{
	{\raggedleft\textsc{#1\expandafter\ifstrequal\expandafter{#2}{}{}{\hspace{6pt}\footnotesize{(#2)}}}\par} % Duration and conditional full time/part time text
	\expandafter\ifstrequal\expandafter{#3}{}{}{{\raggedright\large #3}\\} % Employer
	\expandafter\ifstrequal\expandafter{#4}{}{}{{\raggedright\large\textit{\textbf{#4}}}\\[4pt]} % Job title
	\expandafter\ifstrequal\expandafter{#5}{}{}{#5\\} % Description
}

% Command for entering a separate qualification
\newcommand{\educationentry}[5]{
	\textsc{#1} & \textbf{#2}\\ % Duration and degree
	\expandafter\ifstrequal\expandafter{#3}{}{}{& {\small\textsc{#3}}\\} % Honours, achievements or distinctions (e.g. first class honours)
	\expandafter\ifstrequal\expandafter{#4}{}{}{& #4\\} % Department
	\expandafter\ifstrequal\expandafter{#5}{}{}{& \textit{#5}\\[6pt]} % Institution
}

% Command for entering a separate table row -- used as a generic visual element for any section that uses a two column table
\newcommand{\tableentry}[3]{
	\textsc{#1} & #2\expandafter\ifstrequal\expandafter{#3}{}{\\}{\\[6pt]} % First the heading, then content, then a conditional insertion of whitespace if the third parameter has any content in it
}

% Command for entering a long-form description where there is a title on one line and a paragraph description below it
\newcommand{\longformdescription}[2]{
	\textit{#1}\\[3pt]
	#2\medskip
}

% Command for entering a publication in long-form format
\newcommand{\longformpublication}[1]{
	#1\medskip
}

% Command for entering a publication as a short DOI (digital object identifier) string to the publication, a link is automatically created
\newcommand{\doipublication}[4]{
	#1 & % Year
	\href{http://dx.doi.org/#2}{\expandafter\ifstrequal\expandafter{#3}{firstauthor}{\textbf{doi:#2}}{doi:#2}}% DOI string and if "firstauthor" is entered for the 3rd argument, this makes the DOI string bold indicating a first author publication
	\expandafter\ifstrequal\expandafter{#4}{}{\\}{\\[3pt]} % Conditional insertion of whitespace if the 4th parameter has any content in it
}
